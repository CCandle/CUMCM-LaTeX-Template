\section{公式}
\subsection{行内公式}
行内公式用两个\$包裹起来即可:

示例公式$E=mc^2$
\subsection{单行公式}
行间公式有多种写法,最常用的是equation,带编号,如\autoref{eq:eqsample}
\begin{equation}
    \sum_{i=1}^{n}\frac{1}{i}>\int_{1}^{n}\frac{1}{x}\d x=\ln(n+1)
    \label{eq:eqsample}
\end{equation}

如果公式不需要带编号,则可以简单的用一对\$\$括起来
$$
\int x^2 \d x = \frac{1}{3}x^3
$$

或者像这样
\begin{equation*}
    \int_{-\infty}^{+\infty}\delta(t)=1
\end{equation*}

\subsection{多行公式}
\subsubsection{align}
最常用align*。该环境下每行都会按照\&所在位置对齐。
\begin{align*}
    \int \frac{1}{x\ln x}\d x 
    &= \int \frac{1}{\ln x}\cdot \frac{1}{x}\d x\\
    &= \int \frac{1}{\ln x} \d \ln x\\
    &= \int  \d \ln((\ln x))\\
    &= \ln(\ln x)
\end{align*}

或者带编号,如\autoref{eq:alisample}
\begin{equation}
    \begin{aligned}
        a&=b\\
        c&=d
    \end{aligned}
    \label{eq:alisample}
\end{equation}

不使用align环境的原因是:align环境每一行都会有编号,使用较少。

大括号也是类似的方法生成的
$$
\varepsilon(t)=
\left\{
\begin{aligned}
    1&,t\ge 0\\
    0&,t<0
\end{aligned}
\right.
$$

\subsubsection{gather}
居中对齐环境
\begin{gather*}
    a+b=b+a\\
    a=b=c
\end{gather*}

\subsubsection{array}
array环境类似于表格,通常用于输出数组或更精细的公式排版
$$
\left(
    \begin{array}{cccc}
        a_{00}&a_{01}&a_{02}&a_{03}\\
        a_{10}&a_{11}&a_{12}&a_{13}\\
        a_{20}&a_{21}&a_{22}&a_{23}\\
        a_{30}&a_{31}&a_{32}&a_{33}\\
    \end{array}
\right)
$$