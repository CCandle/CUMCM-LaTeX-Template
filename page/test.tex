\section{样式测试}
\subsection{二级标题}
\subsubsection{三级标题}
正文内容。正文内容正常撰写即可。
默认回车不会分段。

如果需要分段,则需要空一行。

\subsection{中英文测试}
\subsubsection{英文测试}
\blindtext[2]

\subsubsection{中文测试}
这是中文内容测试。

此开卷第一回也。作者自云:因曾历过一番梦幻之后,故将真事隐去,而借"通灵"之说,撰此《石头记》一书也。故曰"甄士隐"云云。但书所记何事何人?自又云:“今风尘碌碌,一事无成,忽念及当日所有之女子,一一细考较去,觉其行止见识,皆出于我之上。何我堂堂须眉,诚不若彼裙钗哉?实愧则有余,悔又无益之大无可如何之日也!当此,则自欲将已往所赖天恩祖德,锦衣纨绔之时,饫甘餍肥之日,背父兄教育之恩,负师友规谈之德,以至今日一技无成,半生潦倒之罪,编述一集,以告天下人:我之罪固不免,然闺阁本自历历有人,万不可因我之不肖,自护己短,一并使其泯灭也。虽今日之茅椽蓬牖,瓦灶绳床,其晨夕风露,阶柳庭花,亦未有妨我之襟怀笔墨者。虽我未学,下笔无,又何妨用假语村言,敷演出一段故事来,亦可使闺阁昭传,复可悦世之目,破人愁闷,不亦宜乎?故曰“贾雨村”云云。

此回凡用“梦”用“幻”等字,是提醒阅者眼目,亦是此书立意本旨。

列位看官:你道此书从何而来?说起根由虽近荒唐,细按则深有趣味。待在下将此来历注明,方使阅者了然不惑。
