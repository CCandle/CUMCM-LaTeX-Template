\documentclass[UTF8,AutoFakeBold,twoside]{XJTUMCM}

% ++++++++++++  文章排版设置  ++++++++++++ %
% ----------  调整页眉页脚样式  ---------- %
\usepackage{fancyhdr}
\pagestyle{fancy}
\fancyhf{}
\setlength{\headheight}{20pt}           % 设置页眉高度
\renewcommand{\headrulewidth}{0mm}      % 设置页眉线
\renewcommand{\footrulewidth}{0mm}      % 设置页脚线
% L-R: 左右
% E-O: 奇偶
\fancyhead[RO]{\leftmark}
\fancyhead[LE]{\titlecontent}
% \fancyhead[CO, CE]{\titlecontent}
\fancyfoot[RO, LE]{\thepage}
% \fancyfoot[C]{\thepage}

% -------------  行间距设置  ------------- %
\usepackage[nodisplayskipstretch]{setspace}
\setstretch{1.2}  % 全文行间距。但是这个跟word的计算方式不一样,目前得凭感觉
\renewcommand{\captionfont}{\linespread{1}} % 题注行间距
\renewcommand{\arraystretch}{1.2}           % 表格行间距

% ------------  可选文章设置  ------------ %
% \usepackage{pdfpages}   % 国赛前两页最好用官方的word导出pdf插入比较好
\usepackage{blindtext}    % 生成无用文本看效果,没用
% \usepackage{sty/matlab} % 自定义的matlab语言关键词,代码高亮用
% \usepackage{longtable}  % 跨页长表格
% \usepackage{tabularx}   % 表格与页同宽所需环境
% \usepackage{booktabs}   % 做三线表会比较需要
% \usepackage{subcaption} % 子图题注
% \usepackage{inconsolata}% linux等宽字体


% ++++++++++++  此处更改标题  ++++++++++++ %
\title{XJTU科技论文模板}


\begin{document}
\bibliographystyle{gbt7714-numerical}
\nocite{*}  % 为了使所有参考文献加载出来。如果在文中已经用\cite标注出文献
            % 并且不需要所有参考文献都加载出来,则注释掉这一行
\maketitle
\begin{abstract}[keyword1][keyword2]
    abstract content
\end{abstract} % 引入摘要内容

% ++++++++++++  此处更改内容  ++++++++++++ %
\section{测试章节}
\subsection{次级标题}
\subsubsection{三级标题}
\blindtext
% 每一章节都在./page 下新建 .tex 文件并\include即可。
% 如果不想自动分页,可以将include改为input

\newpage
\bibliography{doc/bibfile}
\newpage
\begin{appendix}
    \section{代码}
    \lstinputlisting[
        style=python,
        breaklines=true,
        caption={python-sample}]{src/sample.py}
    \lstinputlisting[
        style=matlab,
        breaklines=true,
        caption={matlab-sample}]{src/sample.m}
\end{appendix}                                                              
\end{document}
